% Define my own latex command

%%% Fonts
\newcommand*{\myfont}{\fontfamily{cmr}\selectfont}
\newenvironment{myFont}{\myfont}{\par}
\newenvironment{myCustomFont}{\AlegreyaSans}{}

%%%% Colors
\definecolor{MagSombre}{rgb}{0.5,0.09,0.09}
\definecolor{green_forest}{RGB}{34,139,34}
\definecolor{qqzzqq}{rgb}{0,0.6,0}
\definecolor{ffqqtt}{rgb}{1,0,0.2}
\definecolor{ududff}{rgb}{0.30196078431372547,0.30196078431372547,1}

\definecolor{gris_up}{RGB}{235,235,235}
\definecolor{brun_up}{RGB}{155,137,118}
\definecolor{rouge_up}{RGB}{164,16,42}

\definecolor{vert_lma}{RGB}{1,160,138}
\definecolor{vertf_lma}{RGB}{0,95,80}
\definecolor{bleu_lma}{RGB}{46,69,145}

\definecolor{brun_down}{RGB}{125,109,91}

% Notes in the document
\newcommand\mynotes[1]{\begin{center}
	\vspace{7mm}\textit{\textcolor{rouge_up}{#1}}\vspace{7mm}\end{center}}

%%% Numbering
\setcounter{secnumdepth}{3}

%%% Define a new line spacing
\linespread{1.15}

%%% Config fancyhdr
\pagestyle{fancy}
\fancyhead{}
\fancyfoot{}

\renewcommand{\chaptermark}[1]
	{\markboth{\chaptername ~ \thechapter\ --- #1}{}}
	\renewcommand{\sectionmark}[1]%
	{\markright{\thesection.\ #1}}
\renewcommand{\sectionmark}[1]{\markright{#1}}
\fancyhf{}
	\fancyhead[RO,LE]{\myCustomFont{ \thepage}}
	%\fancyhead[LE]{\myCustomFont{ \leftmark}}
	%\fancyhead[RO]{\myCustomFont{ \rightmark}}
	%\fancyfoot[C]{}
	\renewcommand{\headrule}{\color{gris_up}{\rule{\textwidth}{0.5mm}}}
%	\renewcommand{\headrulewidth}{0.5pt}
%\newcommand{\startChapter}{\vspace{-1.5cm}\minitoc \pagestyle{fancy}}

\fancypagestyle{plain}{%
\fancyfoot[C]{\copyright \\ Page \thepage}
\renewcommand{\headrule}{\color{gris_up}{\rule{\textwidth}{0.5mm}}}
%\renewcommand{\headrulewidth}{0.5pt}
}

%%% Change paragraph appearence        
\titleformat{\paragraph}
{}
{}
{0pt}
{\normalsize\textit{#1}}



%%% Indentation
\titlespacing*{\section}{0pt}{*6}{*2}
\titlespacing*{\subsection}{2em}{*6}{*2}
\titlespacing*{\subsubsection}{4em}{*6}{*2}
\setlength{\parindent}{0.5cm}

%%% raccourcis
\newcommand{\etc}{\textit{etc}}

%%% Hyperref
\hypersetup{
    bookmarks=true,         % show bookmarks bar?
    unicode=false,          % non-Latin characters in Acrobat’s bookmarks
    pdftoolbar=true,        % show Acrobat’s toolbar?
    pdfmenubar=true,        % show Acrobat’s menu?
    pdffitwindow=false,     % window fit to page when opened
    pdfstartview={FitH},    % fits the width of the page to the window
    pdftitle={\Title},    % title
    pdfauthor={\Author},     % author
    pdfsubject={Thèse de modélisation mathématique},   % subject of the document
    pdfcreator={\Author},   % creator of the document
    pdfproducer={\Author}, % producer of the document
    pdfkeywords={Gliome, Modélisation, Cerveau, Substrats, Lactate, Métabolisme énergétique, IRM, SRM, Equations}, % list of keywords
    pdfnewwindow=true,      % links in new PDF window
    colorlinks=true,       % false: boxed links; true: colored links
    linkcolor= MagSombre,
    linkcolor= MagSombre,
    linkbordercolor = white,         
    citecolor= green_forest, % color of links to bibliography
    filecolor=magenta,      % color of file links
    urlcolor=bleu_lma           % color of external links
}


\newenvironment{watermarking}%
{
  % liste des commandes dans l'environnement
  \newcommand{\gf}[1]{\mathbb{F}_{##1}}
  \newcommand{\gftwo}[1]{\mathbb{F}_{2^{##1}}}
  \newcommand{\gfq}[1]{\mathbb{F}_{q^{##1}}}

  \newcommand{\x}{\mathbf{x}}
  \newcommand{\mbf}[1]{\mathbf{##1}}
  \newcommand{\vect}[1]{##1_1, \ldots, ##1_n}
%  \newcommand{\gfq}[1]{\mathbb{F}_{q^{##1}}}
}%
{}

%-------------------------------------------Commandes en supp

%mettre des commentaires
\usepackage{verbatim}



%Pour nettoyer les pages paires entre deux chapitres
\makeatletter
\renewcommand{\cleardoublepage}{%
  \clearpage\fancyhead{}
  \if@twoside
    \ifodd\c@page
      %
    \else
      \hbox{}\newpage
      \if@twocolumn
         \hbox{}\newpage
      \fi
    \fi
  \fi
  
  \fancyhead[RO,LE]{\myCustomFont{ \thepage}}}
\makeatother

%profondeur table des matières
\setcounter{tocdepth}{1}     % Dans la table des matieres 
\setcounter{secnumdepth}{4}  % Avec un numero.


%Mes alinéas
\newcommand{\1}{\displaystyle{\mathds{1}}}
\newcommand{\Li}{\mathscr{L}}
\newcommand{\R}{\mathbb{R}}
\newcommand{\hs}{\hspace{0.02cm}}
\newcommand*{\norm}[1]{\left\lVert#1\right\rVert} % Norme
\newcommand*{\abs}[1]{\left \vert#1 \right \vert} % Valeur absolue
\newenvironment{pushright}{\begin{itemize}\item[\hspace{12pt}]}{\end{itemize}}
\newcommand{\di}{\mbox{d}}
\newcommand{\fat}{$\forall~t \in \mathbb{R}^+$}  
\newcommand{\banana}{{\color{red}\Huge \begin{center}MANQUE IMAGE !\end{center}}}
\newcommand{\bananay}{{\color{red}\Huge \begin{center}MANQUE TEXTE !\end{center}}}


%fonctions avec barre
\newcommand{\fnc}[5]{ \begin{array}{c|ccc} #1: &#2&\longrightarrow &#3\\ &#4&\longmapsto &#5\end{array}} 

%cas
\newcommand{\cas}[4]{#1=\left\{
\begin{array}{rl}
#2 &\text{ si }#3 \\
#4 &\text{ sinon.}
\end{array}
\right.}

%casf
\newcommand{\casf}[3]{\left\{
\begin{array}{rl}
#1 &\text{ si }#2 \\
#3 &\text{ sinon.}
\end{array}
\right.}

%casf2
\newcommand{\casff}[5]{\left\{
\begin{array}{rl}
#1 &\text{ si }#2 \\
#3 &\text{ si }#4 \\
#5 &\text{ sinon.}
\end{array}
\right.}

%cas2
\newcommand{\cass}[5]{#1=\left\{
\begin{array}{rl}
#2 &\text{ si }#3 \\
#4 &\text{ si }#5
\end{array}
\right.}

\newcommand{\fnct}[5]{#1 \left\{
\begin{array}{rl}
#2&\longrightarrow #3\\
#4&\longmapsto  #5
\end{array}
\right.}   


%Diminuer taille titre
\captionsetup{font=small,labelfont={color=MagSombre,bf}}

%%% Create chapter header

\makeatletter
\renewcommand\tableofcontents{%
  \null\hfill\textbf{\Large\color{MagSombre}\contentsname} \hfill\null\par
  \@mkboth{\contentsname}{\contentsname}%
  \@starttoc{toc}%
}
\makeatother


\titleformat{\section}[frame] 
  {\setlength\fboxrule{3pt}\bfseries\fontsize{13.8}{14}\selectfont\color{gris_up}}
  {}{11pt}
  {\color{MagSombre}$\;\;$\thesection .\hskip 0.7em#1}

	